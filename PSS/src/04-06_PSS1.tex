\documentclass[boxes,sansserif]{rutgers_hw}
\usepackage{rutgers}

\author{Ravi D'Elia}
\netid{rjd280}

\assignment{PSS-1} % Enter the assignment name
\date{2023-04-06} % Replace with due date
\institution{Rutgers University} % Enter your university

\begin{document}
\maketitle
\setlength\parindent{0pt}

We seek to prove that \textbf{IV} implies \textbf{I}.

Let $V$ be a vector space, $\left\{ W_i \right\} \subset V$, $\gamma_i$ be a basis of $W_i$,
and $\cup_i \gamma_i$ is a basis of all of $V$.

First we must show that arbitrary sums of $W_i$s intersect only at $0$. Suppose that
$\exists v \in W_i$ such that $v = \sum w_j$ for $w_j \in W_j$, $j \in 1..(i-1)$. But then
we have
\begin{displaymath}
  \sum_{j = 1} a_j \gamma_i^j = \sum b_k \gamma_1^k + ... + \sum b_k \gamma_j^k
\end{displaymath}
But then $v$ can be uniquely specified as linear combinations of all $\gamma_i^j$ in more
than one way, contradicting that it is a basis. So there can be no non-zero intersections.

Now we must show that $v \in V$ can be represented as a sum of elements of each $W_i$. This
is trivial. As $\gamma_i^j$ is a basis, we have
\begin{displaymath}
  v = \sum a_i^j \gamma_i^j = \sum_i \sum_j a_i^j \gamma_i^j = \sum_i w_i
\end{displaymath}
for $w_i \in W_i$. 

\end{document}