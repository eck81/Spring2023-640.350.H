\documentclass[main.tex]{subfiles}

\begin{document}
    \chapter{Properties Of The Null Space And Range}

    Recall that a linear transformation between vector spaces $V$ and $W$ over a field $F$, is a function $T: V\to W$ such that 
    \begin{enumerate}
        \item $T(x+y) = T(x) + T(y)$ for all $x,y\in V$
        \item $T(kx) = kT(x)$ for all $x\in V$ and $k\in F$
    \end{enumerate}
    Mainly, a linear transformation preserves linear combinations between vectors. So we have $T(ax+by) = aT(x) + bT(y)$. Here are some examples of linear transformations. 
    \begin{itemize}
        \item From linear algebra: matrix-vector products are linear transformations. So $A(u+v) = Au + Av$ and $A(cu) = c(Au)$. 

        \item From calculus: derivatives and anti-derivatives are linear transformations. 

        \item From geometry: rotations and reflections are linear transformations.
    \end{itemize}

    \section{Null Spaces \& Ranges}
    Two important spaces associated with a linear transformation are the null space (kernel) and range. Two important facts about these subspaces are given below 
    \begin{itemize}
        \item The null space of $T$ is a subspace of $V$ while the range is a subspace of $W$. 
        \item The range of a linear transformation is $\Range T = \Span \{T(v_1), T(v_2), ..., T(v_n)\}$ where $\{v_1, v_2, ..., v_n\}$ is a basis of $V$. In other words, we only need to know the images of the basis vectors to generate the range of the transformation.
    \end{itemize}

    While the null space and range seem unrelated, they are actually somewhat connected. This will be the goal of today's lecture. We will introduce two new terms in this part. 
    \begin{defn}{Rank \& Nullity}{}
        For a linear transformation $T: V\to W$, we have 
        \begin{enumerate}
            \item The rank of a transformation is the dimension of its range. 
            \item The nullity of a transformation is the dimension of its null space.
        \end{enumerate}
    \end{defn}
    With this in mind, we now present the following theorem, which details a very simple, but incredibly useful, relationship between the two terms. 
    \begin{thrm}{}{}
        Let $T:V\to W$ be an arbitrary linear transformation where $V$ and $W$ are finite. Then we have that 
        \begin{equation}
            \rank T + \nullity T = \Dim V
        \end{equation}
    \end{thrm}
    Before presenting a formal proof, we will present a concrete example. 
    \begin{example}{}{}
        Consider a linear transformation $T: \R^3 \to \R^2$ given by the rule $T(x_1, x_2, x_3) = (x_1 - x_2, 2x_3)$. The null space is given by $x_3=0$ and $x_1 = x_2$. We can see that the dimension is 1 since a basis for the null space is $\{(1, 1, 0)\}$. \bigbreak 

        We claim that the range of the transformation is all of $\R^2$. For arbitrary $(y_1, y_2)$, we can argue that there is always a preimage of this vector in $\R^3$. This can be done by solving a linear system. 
    \end{example}

    So here is a proof of the theorem 
    \begin{proof}
        Assume $\Dim V = n$ and $\Dim (\Null T) = k$. Note that by a previous theorem, we know that $k\leq n$. To find the rank of $T$, we will generate a basis for the range. Let $\{v_1, v_2, ..., v_k\}$ be a basis for $\Null T$. Now, we can extend this basis to $\{v_1, ..., v_k, v_{k+1}, v_{k+2}, ..., v_n\}$, which is a basis of $V$. Note that there are $n-k$ vectors that were added to the basis. Now consider the vectors $\{T(v_{k+1}), T(v_{k+2}), ..., T(v_n)\}$. Now we claim that this set is a basis for the range of $T$. \par 

        So first we show that $\Range T = \Span \{T(v_1), T(v_2), ..., T(v_k), T(v_{k+1}, ..., T(v_n)\}$. But by choice, we have $\{v_1, ..., v_k\}$ is a basis for the null space. So we get $\Range T = \Span \{T(v_{k+1}, ..., T(v_n)\}$, as desired. \par 

        Now we need to show linear independence. So assume $\sum_{k+1}^n b_iT(v_i) = 0$. We will show that $b_i = 0$. Since $T$ is linear, we have 
        \begin{equation}
            \sum_{k+1}^n b_i T(v_i) = T \left( \sum_{k+1}^n b_i v_i \right) = 0
        \end{equation}
        So we have that $\sum b_iv_i$ is in the null space of $T$. So we observe that there should be a way to represent this vector as a linear combination of the basis vectors of the null space. This implies 
        \begin{equation}
            \sum_{i=k+1}^n b_iv_i - \sum_{i=1}^k c_iv_i=0
        \end{equation}
        And since basis vectors are linearly independent, we have $b_i = 0$ and $c_i = 0$
    \end{proof}

    \section{Injections \& Surjections}
    Recall that we have the definitions of one-to-one functions and onto functions from Math 300. 
    \begin{defn}{Injective Functions \& Surjective Transformations}
        Let $T: V\to W$ be an arbitrary linear transformation. Then, we say $T$ is 
        \begin{enumerate}
            \item \textit{One-to-one} (or injective) if $T(v_1) = T(v_2)$ implies $v_1 = v_2$ for all $v_1, v_2 \in V$

            \item \textit{Onto} (or surjective) if for all $w\in W$, there exists $v\in V$ such that $T(v) = w$
        \end{enumerate}
    \end{defn}

    We can use the null space and range to determine whether or not a transformation is one-to-one or onto. The results are stated in the following theorems. 
    \begin{thrm}{}{}
        A linear transformation $T: V\to W$ is one-to-one if and only if $\Null T = \{0\}$. 
    \end{thrm}
    \begin{proof}
        Proof of $(\implies)$: assume $T$ is one-to-one. Then assume $T(x) = 0$. But from the definition of a linear transformation, we have that $T(0)=0$. Since $T$ is one-to-one, we have $x=0$ and so $\Null T = \{0\}$. \bigbreak 

        \noindent Proof of $(\impliedby)$: assume $\Null T = \{0\}$. Assume that $T(x) = T(y)$. Then, observe that we have 
        \begin{align*}
            T(x) &= T(y) \\
            T(x) - T(y) &= 0 \\
            T(x - y) &= 0 
        \end{align*}
        Then, we have that $x-y\in \Null T$. But the only vector in the null space is $0$. So we must have $x-y=0$. It follows that $x=y$.
    \end{proof}

    From this fact, we have a simple corollary. 
    \begin{cor}{}{}
        Let $T: V\to W$ be a linear transformation with $V$ and $W$ finite and with the same dimension. Then the following are equivalent. 
        \begin{enumerate}
            \item $T$ is one-to-one 
            \item $T$ is onto 
            \item $\Range T = W$
        \end{enumerate}
    \end{cor}
    \begin{proof}
        Proof of $(1)\implies (3)$: if $T$ is one-to-one so $\Null T = \{0\}$. Therefore, $\nullity T = 0$ and so $\rank T = n$. By the rank-nullity theorem, we have that $\Dim V = n = \Dim W$. Now observe that by definition, $\rank T = n = \Dim (\Range T)$. Now observe that $\Range T$ is a subspace of $W$. By a previous result, we have $\Range T = W$, as desired. \bigbreak

        \noindent Proof that $(2) \implies (3)$: assume that $T$ is onto. 
    \end{proof}

    Here is one last theorem, that will be used to connect this topic to the next section. 
    \begin{thrm}{}{}
        Let $\{v_1, v_2, ..., v_n\}$ be a basis of $V$. Let $\{w_1, ..., w_n\}$ be $n$ vectors in $W$ (they don't have to be distinct). Then, there exists a unique linear transformation $T: V\to W$ such that $T(v_i) = w_i$.
    \end{thrm}
\end{document}